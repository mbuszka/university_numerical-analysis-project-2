\documentclass[11pt,wide]{mwart}
\usepackage[utf8]{inputenc} 
\usepackage[OT4,plmath]{polski}
\usepackage{graphicx}
\usepackage{caption}
\usepackage{subcaption}
\usepackage{epstopdf}
\usepackage{alltt}
\usepackage[section]{placeins}
\usepackage{graphicx}

\usepackage{amsmath,amssymb,amsfonts,amsthm,mathtools}


\usepackage{bbm}
\usepackage{hyperref}
\usepackage{url}

\usepackage{comment}

\date{Wrocław, \today}
\title{\LARGE\textbf{Pracownia z analizy numerycznej}
  \\Sprawozdanie do zadania \textbf{P2.11.}}

\author{Maciej Buszka}

\newtheorem{tw}{Twierdzenie}
\newtheorem{alg}{Algorytm}
\newtheorem{defn}{Definicja}

\begin{document}
\maketitle
\thispagestyle{empty}
\tableofcontents

\section{Wstęp}
\section{Interpolacja funkcją sklejaną III stopnia}

\subsection{Definicje}

\begin{defn}

Niech punkty $ (t_0, y_0) , \ldots, (t_n, y_n) $ będą węzłami interpolacji, wtedy funkcją sklejaną III stopnia nazywamy funkcję $ s: [t_0, t_n] \rightarrow \mathbb{R} $ spełniającą następujące warunki:
\begin{enumerate}
\item $ s(x_i) = y_i $
\item W każdym z przedziałów $ [t_{i-1}, t_i ) $, $ i = 1, \ldots, n $ $ s = s_i $  jest wielomianem trzeciego stopnia.
\item s ma ciągłą pierwszą i drugą pochodną (w przedziale $ [t_0, t_n] $).
\end{enumerate}

\end{defn}

Zauważmy, że taka definicja funkcji interpolującej układ punktów $ (t_0, y_0) , \ldots, (t_n, y_n) $ generuje następujące warunki, które $ s $ musi spełniać:

\begin{align}
	s(t_i) &= y_i               		&\text{dla }& i = 0, \ldots, n \\
	s_i(t_{i}) &= y_i = s_{i+1}(t_{i}) 	&\text{dla }& i = 1, \ldots, n-1 \label{eq:scont}\\
	s_i'(t_{i}) &= s_{i+1}'(t_i) 		&\text{dla }& i = 1, \ldots, n-1 \label{eq:dscont}\\
	s_i''(t_{i}) &= s_{i+1}''(t_i) 	    &\text{dla }& i = 1, \ldots, n-1 \label{eq:ddscont}
\end{align}
Łącznie równania te dają $ 4n - 2 $ warunki, a każdy z $ n $ wielomianów $ s_i $ ma $ 4 $ współczynniki, pozostają zatem jeszcze dwa dodatkowe warunki, które można zadać funkcji $ s $ w pozostałej części sprawozdania rozważana będzie okresowa funkcja sklejana trzeciego stopnia.
\begin{defn} Jeżeli węzły $ (t_0, y_0) $ i $ (t_n, y_n) $ są sobie równe, to funkcję sklejaną $ s $ interpolującą węzły $ (t_0, y_0) , \ldots, (t_n, y_n) $ nazywamy okresową, jeżeli
\begin{align}
s'(t_0) &= s'(t_n) \\
s''(t_0) &= s''(t_n)
\end{align}
\end{defn}

\subsection{Wyprowadzenie układu równań}

Niech dane będą punkty $ (t_0, y_0) , \ldots, (t_n, y_n) $. Wprowadźmy oznaczenie $ k_i = s''(t_i) $ i rozpatrzmy $ s_i $ w przedziale $ [t_i, t_{i+1} ) $. Z definicji $ s_i $ jest wielomianem trzeciego stopnia, zatem $ s_i'' $ jest funkcją liniową. Z warunku \eqref{eq:ddscont} wynika zatem, że
\begin{equation}
	as
\end{equation}

\section{Krzywa parametryczna}
\section{Wyniki eksperymentów}
\section{Analiza wyników}
\section{Wnioski}

\begin{thebibliography}{99}

\bibitem{kincaid} David Kincaid, Ward Cheney, przekł.~Stefan Paszkowski,
\emph{Analiza numeryczna},
Warszawa, WNT, 2006.

\bibitem{dahlquist} Germund Dahlquist, \r{A}ke Bj\"{o}rck,
\emph{Numerical Methods in Scientific Computing Volume I}
SIAM, 2008

\bibitem{bjorck} \r{A}ke Bj\"{o}rck, Germund Dahlquist, przekł.~Stefan Paszkowski
\emph{Metody Numeryczne},
Warszawa, PWN, 1987

\bibitem{eigen} \r{A}ke Bj\"{o}rck, Gene H. Golub
\emph{Eigenproblems for Matrices Associated with Periodic Boundary Conditions}

\bibitem{epstein} M. P. Epstein, 
\emph{On the Influence of Parametrization in Parametric Interpolation},
SIAM Journal on Numerical Analysis Volume 13, Issue 2, 1976


\end{thebibliography}

\end{document}